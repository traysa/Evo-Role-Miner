\newpage
\subsection{Phase Two - Basic Setup}
\label{sec:phase2}
In the second phase the fitness functions $F_{basic}^{min}$ \eqref{eq:FBasicMin} and $F_{edge}^{min}$ \eqref{eq:FEdgeMin} are tested on the small synthetic dataset 1 and the healthcare dataset. The setup for the experiments can be seen in table \ref{tab:setup2}.
\begin{table}[H]
    \centering
    \begin{tabular}{|l|l|}
        \hline
        \rowcolor{myGray} 
        \textbf{Parameter}              & \textbf{Value}    \\ \hline
        Generations                     & 1000              \\ \hline
        Population                      & 1000              \\ \hline
        CXPB                            & 0.25              \\ \hline
        MUTPB                           & 0.25              \\ \hline
        MUTPB-Type1: Add role           & 0.25              \\ \hline
        MUTPB-Type2: Add User           & 0.25              \\ \hline
        MUTPB-Type3: Add Permission     & 0.25              \\ \hline
        MUTPB-Type4: Remove Role        & 0.25              \\ \hline
        MUTPB-Type5: Remove User        & 0.25              \\ \hline
        MUTPB-Type6: Remove Permission  & 0.25              \\ \hline
        Tournament size                 & 2                 \\ \hline
        Local optimization              & True              \\ \hline
        Weights for Fitness Function    & 1.0, 1.0, 1.0     \\ \hline
    \end{tabular}
    \caption{EXPERIMENT 2 setup}
    \label{tab:setup2}
\end{table}
The results of the experiments can be seen in experiment 2 in table \ref{tab:results_exp2_dataset1} and experiment 3 in table \ref{tab:results_exp2_healthcare}. The fitness graphs can be seen in the Appendix \ref{sec:AppendixB}. In all experiments the fitness is improving over time. The fitness functions look like a hyperbola, where the improvement is strong in the first generations and less in the last generations.\\
For the experiments with fitness function $F_{basic}^{min}$ on the synthetic dataset 1 (see experiment 2a in table \ref{tab:results_exp2_dataset1}), the average minimum number of roles in the results are 2, which is according to the objective of minimizing roles in $F_{basic}^{min}$ a good result. But the number of roles in the synthetic dataset 1 is known to be 4. The discovered small amount of roles has the drawback that the other objectives are harder to reach. The number of confidentiality violations has not reached zero in any of the 10 experiments, where it is known that a solution exists with no violations when the role count is 4. Also the number of roles for the healthcare dataset (see experiment 3a in table \ref{tab:results_exp2_healthcare}) seem to be very low compared to results in other papers where the lowest role count is 14 \cite{Ene}\cite{Molloy:2009:ERM:1542207.1542224}. The number of violations for the healthcare dataset are not optimal, in the sense that when the according role model is applied some users will gain access rights and others will loose some. If this is acceptable is hard to judge since the details of the healthcare dataset are unknown, e.g. how critical certain permissions are and how much noise is given.\\
The experiments with fitness function $F_{edge}^{min}$ show a slightly higher amount of the average minimum role count (see experiment 2b in table \ref{tab:results_exp2_dataset1} and 3b in table \ref{tab:results_exp2_healthcare}). Comparing the minimum amount of roles of each experiment in experiment 2a and 2b with an unpaired T-Tests confirm that the difference is statistically significant. The same is confirmed for the experiments in experiment 3a and 3b. This seem reasonable, since the fitness function $F_{edge}^{min}$ tries to minimize $|UA|$ and $|PA|$ and only has an indirect focus on minimizing the role count $|R|$.\\
A fitness optimal solution of dataset 1 is not reached in the ten experiments with fitness functions $F_{basic}^{min}$ and is reached once in ten experiments with fitness functions $F_{edge}^{min}$ as seen in Figure \ref{fig:exp2edge_RM}. For the healthcare dataset no fitness optimal solution is reached in any experiment.\\
There are several elements in the EA, which can influence the number of roles. One element is the mutation and the probability for role removal. The fitness function also has an big impact. While in $F_{basic}^{min}$ it is obvious that the weight for the Role count $|R|$ has an impact, the tests in phase 1 showed that also the number of $|UA|$ and $|PA|$ have an indirect influence on the role count. Therefore the weight for $|UA|$ + $|PA|$ in $F_{edge}^{min}$ can influence the role count. Furthermore also the violation count ($G_{conf}$ and $G_{accs}$) have an indirect impact on the role count, which was also found in the first phase of experiments (see section \ref{sec:phase1}). A third element, which can lead to reduction of roles is the local optimization after a mutation and a crossover (see section \ref{sec:localOptimization}).\\
To relax this strong bias towards a small role count, the probability for role removal is set lower and the probability for adding new roles is set higher for the next experiments. Furthermore the weight $w_1$ in the fitness functions $F_{basic}^{min}$ and $F_{edge}^{min}$ is adjusted. \hl{The local optimization is turned off.}

\begin{figure}[H]
    \centering
    \includegraphics[scale=0.37, trim=4cm 2cm 4cm 2cm, clip=true]{./Figures/exp2edge_RM}
    \caption{EXPERIMENT 2b: Example of fitness-optimal solution role model resulting of EvoRoleMiner with Fitness function $F_{edge}^{min}$ on synthetic dataset 1 with setup in table \ref{tab:setup2}. From u.l. to l.r.: User-Role Matrix (Rows: Users, Columns: Roles), Role-Permission Matrix (Rows: Roles, Columns: Permissions), Resulting User-Permission Matrix (Rows: Users, Columns: Permissions), Original User-Permission Matrix from Input (Rows: Users, Columns: Permissions). A blue box stands for an assignment.}
    \label{fig:exp2edge_RM}
\end{figure}

\begin{landscape}
    \begin{table}
        \centering
        \begin{tabular}{|l|l|c|c|c|c|c|c|c|c|}
            \hline
            \rowcolor{myGray} 
            \textbf{Experiment} & \textbf{Fitness Function} & \textbf{Fitness} & \textbf{\# $G_{conf}$} & \textbf{\# $G_{accs}$} & \textbf{\# Roles} & \textbf{\# UR} & \textbf{\# RP} & \textbf{INT} & \textbf{Time (in sec)}\\ \hline
            2a & $F_{basic}^{min}$ \eqref{eq:FBasicMin}   &   0.38   &   2.8   &   0   &   2   &   8.9   &   7.9   &  1   &   336\\ \hline
            2b & $F_{edge}^{min}$ \eqref{eq:FEdgeMin}   &    0.15    &   2.8   &   0   &   3.1   &   9.6   &   11.1   &   1   &   343\\ \hline
            
            4a & $F_{basic}^{min}$ \eqref{eq:FBasicMin}   &   0.08   &   0   &   0   &   3.9   &   10   &   13.3   &   1   & 372\\ \hline
            4b & $F_{edge}^{min}$ \eqref{eq:FEdgeMin}   &   0.05   &   0.2   &   0   &   3.9   &   11.4   &   10.8   &   0.998   & 371\\ \hline
            
            x & $F_{basic\_INT}^{min}$ \eqref{eq:FBasicMin_INT}   &   0.38  &   2.8   &   0   &   2   &   8.5   &   8   &   1   &   421\\ \hline
            x & $F_{edge\_INT}^{min}$ \eqref{eq:FEdgeMin_INT}   &   0.14   &   1.9   &   0   &   3.2   &   9.9   &   11   &   1   &   530\\ \hline
        \end{tabular}
        \caption{Results of experiments for the small synthgetic Dataset 1. The values for Fitness, Confidentiality violations, Availability violations, Roles, User-Role-Assignments and Role-Permission assignments are the average minimum in the last Generation out of ten experiments. The values for Interpretability are the average maximum in the last Generation out of ten experiments. The time is the average runtime in seconds of one experiment.}
        \label{tab:results_exp2_dataset1}
    \end{table}
    \begin{table}
        \centering
        \begin{tabular}{|l|l|c|c|c|c|c|c|c|c|}
            \hline
            \rowcolor{myGray} 
            \textbf{Experiment} & \textbf{Fitness Function} & \textbf{Fitness} & \textbf{\# $G_{conf}$} & \textbf{\# $G_{accs}$} & \textbf{\# Roles} & \textbf{\# UR} & \textbf{\# RP} & \textbf{INT} & \textbf{Time (in sec)}\\ \hline
            3a & $F_{basic}^{min}$ \eqref{eq:FBasicMin_INT}   &   0.13   &   3.6   &   48.4   &   2   &   54.8   &   60.8   &   -   & 664\\ \hline
            3b & $F_{edge}^{min}$ \eqref{eq:FEdgeMin_INT}   &   0.12   &   4.5   &   52.5   &   3.3   &   62.3   &   56.6   &   -   & 775\\ \hline
            5a & $F_{basic}^{min}$ \eqref{eq:FBasicMin_INT}   &   0.09   &   1.3   &   43.6   &   8.5   &   153.5   &   154.4   &   -   & 1329\\ \hline
            5b & $F_{edge}^{min}$ \eqref{eq:FEdgeMin_INT}   &   0.05   &   2.1   &   18.1   &   14.8   &   155   &   168.6   &   -   & 1554\\ \hline
        \end{tabular}
        \caption{Results of experiments for the healthcare dataset. The values for Fitness, Confidentiality violations, Availability violations, Roles, User-Role-Assignments and Role-Permission assignments are the average minimum in the last Generation out of ten experiments. The time is the average runtime in seconds of one experiment.}
        \label{tab:results_exp2_healthcare}
    \end{table}
\end{landscape}