\newpage
\subsection{Phase Four - Interpretability Objective}
\label{sec:phase4}
When looking back on the solution with an optimal fitness found in one of the experiments in Phase 2 for dataset 1, it can be noted that the matrix decomposition of the resulting UPA (see Figure \ref{fig:exp2edge_RM}) looks different than the matrices dataset 1 has been constructed on (see Figure \ref{fig:dataset1}). Other matrix decompositions of the UPA of the synthetic dataset 1 can be seen in the Appendix in \ref{fig:exp4aBasic_RM} and \ref{fig:exp4bEdge_RM}. It shows that several solutions for a matrix decomposition can exist. To distinguish them another measure is needed. One reason why bottom-up role mining is criticized, is that a solution might be found based on given technical data (user-permission-assignments), but it is not necessary feasible for the business. Therefore the Interpretability objective and the fitness functions
$F_{basic\_INT}^{min}$ \eqref{eq:FEdgeMin_INT} and $F_{edge\_INT}^{min}$ \eqref{eq:FEdgeMin_INT} are observed in this phase.\\
For the experiments the setup in table \ref{tab:setup3} is taken and executed on the synthetic dataset 1. The weight for the interpretability measure is set 1.0. The results can be seen in Experiment 6 in Table \label{tab:results_exp2_dataset1}.\\
The same experiments have been tried to execute on the healthcare dataset with the user attribute information provided by Xu \& Stoller\cite{Xu}. The experiments have been aborted after the fitness calculation of the first generation could not be finished in a decent time. There are 20 attributes for the users in the healthcare dataset. For the interpretability measure rules are generated, where $2^X-1$ combinations of X attributes are created. While for the synthetic dataset 1 with 3 user attributes considered seven combinations are created and tested, the healthcare dataset requires 1048575 combinations for 20 attributes. It rule induction gets too expensive to finish in reasonable time.