\newpage
\chapter{Future Work}
\label{sec:futureWork}
% What could you not answer? Suggestions for further research
The experiments in this thesis have only been executed on two small datasets. To get a better understanding on how to set parameters in the suggested EA and MOEA several datasets of different dimensions (user and permission size) and density (number of assignments) should be tested. Also applying evolutionary algorithms on noisy access control policies has to be investigated. It is very likely that given access control policies have a lot of over-entitlements and some under-entitlements. It would be interesting to see how an EA can reveal these false entitlements.

There are several parts in the EA and MOEA, which can be further improved, for example to develop more intelligent variation operators. A different representation of individuals might reveal new opportunities. The approach in this thesis builds strongly on the related boolean matrix decomposition problem (see section \ref{sec:BMD-Problem}). A different approach and representation might be more suitable. For example a representation, which is based on user attributes to include the interpretability of roles into the role mining.

A novelty search could counteract on limited diversity in a population. The basic idea is to explicitly reward individuals, which are diverging from previous individuals. By this a greater search space can be reached more efficiently. The measure for novelty could be an additional objective in the fitness function.

A motivation for the approach was also to incorporate human feedback in the EA. An interactive evolutionary computation (IEC)\cite{949485} could be an interesting topic for further research. Domain experts could navigate the algorithm to an optimal result, accepted by the business, by selecting the individuals for reproduction. Such an approach might consider a different representation of individuals, since it is difficult for a human to evaluate a whole role model. The evaluation of single roles on the other hand can be accomplished by experts.

Another idea is the usage of Co-Evolution. In Co-Evolution several separate populations are influencing the fitness of the individuals of each other. This idea is derived by co-evolution in biology, where the evolution of a species can influence the evolution of another species and vice versa. Instead of that individuals represent RBAC models, the individuals represent roles. The motivation behind this approach is to evolve candidate roles instead of a complete role model. For example each population consists of roles as individuals. Each population might only consider a specific segment of the access control policy, e.g. only users of a department or only permissions of an application. Roles from each population are then getting into a trial, where they should form a role model, which then gets evaluated. This technique is inspired by the Symbiotic Adaptive Neuro-Evolution (SANE) and Enforced Sub-Populations Method (ESP) approach in Gomez \& Miikkulainen\cite{Gomez:1999}.

To find a measure for the comprehension of a role model or the interpretability of roles is an interesting topic. The initial ideas suggested in section \ref{sec:meaningfulness} could be investigated further. A more detailed research on how such an measure could be integrated into a role mining algorithm could be done.

    