\newpage
\chapter{Discussion}
\label{sec:discussion}
\hl{SECTION UNDER CONSTRUCTION}\\
\iffalse
\hl{adequacy, efficiency, productiveness, effectiveness (choose your criteria, state them clearly and justify them)
be careful that you are using a fair measure, and that you are actually measuring what you claim to be measuring
if comparing with previous techniques those techniques must be described in Chapter 2
be honest in evaluation
admit weaknesses
\\\\
The Results and Discussion portion of the thesis.  These two components should remain separated with the appropriate headings within this chapter, as they both serve a different function.  The results portion only presents the hard data without any accompanying analysis or interpretation.  This section should include, where possible, a visual representation of the data, such as in charts, graphs, or tables.  Each figure should have a brief description associated with it and clearly marked labels.  The results of all statistical analyses should be presented, such that the reader has enough information to determine reliability, validity, and the statistical significance of the relationships among variables.   This section should also be clearly organized by subheadings.
\\\\
Which could be the system you made and the reasons for various design decisions, what your interview objects said, observations of people using a computer system, stories of a development process, numeral data from a questionnaire, etc. 
The discussion of the findings can be included in these chapters, or the discussion can be put  in a separate chapter. 
The issues from the theory chapter (chapter 2) should be discussed here.}
\fi

\section{Answer to the research question}
% shortly summarize your question, the core argument and main results, always with respect to your question
The thesis question of \hl{how the role mining problem can be solved efficiently with an evolutionary computation approach} has been investigated by analyzing the domain and providing an EA and three implementations of  a MOEA. Several different fitness functions are exchangeable and several parameters can be set in the implementation and have been tested.

The results show that an EA and an MOEA can find solutions for the role mining problems on small access policy configurations. The tested MOEAs perform better than the EA. The MOEA based on NSGA-II$R$\cite{Fortin:2013} performs better than the MOEA based on NSGA-II\cite{Deb}. The MOEA based on NSGA-II$R$ is extended with a stochastic version of Pareto dominance, which allows to set weights for the objectives\cite{clune2013evolutionary}. The results show that the MOEA with weights contributes the search by relaxing the objective of role model complexity.

\section{Contribution to current research}
% What are the theoretical and empirical implications of your results for current research? What have we learnt?
This thesis contributes with an analysis of the resulting role models of using an EA for Role Mining. Furthermore different MOEAs are applied for Role Mining and show more promising results than using an EA with a scalarization of multiple objectives in one fitness function. The EA and the MOEAs are tested on one of the real datasets commonly used by role mining research, which makes the result more comparable to other work. Furthermore two interpretability measures for roles based on clustering and on classification are suggested, where only the later approach is implemented and tested.

\section{Reflection}
% Critically reflect upon your procedure

\section{Future research}
% What could you not answer? Suggestions for further research

\begin{itemize}
	
	\item Saenko \& Kotenko\\
	The EA for role mining presented in the paper of Saenko \& Kotenko lacks information to re-build the EA. These gaps have tried to be filled in this thesis.	The results in Saenko \& Kotenko are based on artificial datasets and only measure how many generations are needed to find a role model, which describes the given $UPA$. The computation time achieved for different datasets of different dimensions stated in the paper could not be achieved by the approach of this thesis. This might be due to the data, available computation resources or the implementation. In this thesis' approach also a small real dataset, commonly used in role mining research, is tested on the single-objective EA "Evo-RoleMiner", which shall reconstruct the implementation of Saenko \& Kotenko. After re-adjusting parameters several times a decent role model can be found with the Evo-RoleMiner. However, there exists little knowledge on how to choose the weight given only the access policy configuration as input data. The density of assignments in the given access policy configuration might help in the decision, but there is no concrete rule existing. For example, the weight for the number of roles in the fitness function $F_{basic}^{Min}$ had to be set very low or even to a negative value (which encourages a maximization of role numbers) in the Evo-RoleMiner for the synthetic dataset to be able to find a role model, which does not violate the original access configuration policy. One can argue that evolutionary algorithms are not made for finding the one perfect solution.
	
	\item Multi-objective EA solution\\
	The multi-objective EA "Evo-RoleMiner$M$" on the other hand relaxes the challenges of setting the right parameters, by 
	
    \item Real datasets\\
    The experiments in this thesis were mainly executed on a synthetic dataset and on the public available healthcare dataset often used in Role Mining research. Since the used datasets are very small, visualizations of resulting rolemodels can be used to support argumentations. But they might not reflect cases, which enterprises often have to face. Enterprises can have thousands of users and millions of permissions, which might can make digestible in smaller chunks, but will be still bigger than the datasets used in the experiments. \hl{Experiment X showed how quick a bigger dataset rises in computation complexity.}
    
    \hl{Furthermore real datasets contain noise, where overentitlements are more likely than underentitlements. The evolutionary computation approach might not identify the noise.}
    
    \item Role model state vs. roles\\
    In the approach chosen a complete role model state should be generated from scratch. 
    
    An probabilistic approach to generate candidate roles might be more accepted by the business, since single roles are rated and can be chosen. On the otherhand candidate roles are generated with the dependency of other roles.
    
    \item Interpretability vs. Minimal assignments\\
    
    \item Interpretability measure
    Individuals might have the same Fitness and even re-created the original UP-Matrix, but it does not mean that the role model is the same; Additional attributes, like user attributes, could help by e.g. introducing an Interpretability measure to distinguish the role models.
    
    The suggested interpretability measure scales badly the more attributes are considered. \hl{In experiment X the healthcare dataset and the synthetic user attribute information provided by Xu \& Stoller the experiment has been aborted after the initial fitness evaluation has not been finished within X hours.} It is though questionable if user information of 20 attributes is considered in companies. Talking to experts in praxis, the attributes used for hybrid role mining are around a handful, like e.g. employee type, department, costcenter, job function and less often the location. Also, the information of user attributes is often not given or incomplete.
    
    Rules, which might have lead to a role, might not be identified again as the rule, the role was created on
     
    In praxis the resulting roles or rolemodels from bottom-up role mining are analysed and optimized in a post-process, where business information is used as a measure. In this thesis' approach the business information is included in the role mining computation.
    
    By measuring the role interpretability in role models with the help of certain user attributes will mostly reveal roles, which are interpretable by these user attributes. Other roles in the role model will be created to cover the access policy configurations, which ar not covered by the roles with high interpretability according to the chosen user attributes.
    
    The motivation for generating roles, which are interpretable by user attributes, is to predict the roles a user shall have. Furthermore the interpretable roles in RBAC can be easily migrated to ABAC.
    
    \item Min-Noise RMP\\
    The Evo-RoleMiner and the Evo-RoleMiner$M$ provide the option of only calculating role models of a fixed role count, which is anticipated by the Min-Noise RMP. In this case, an other representation of individuals might be more suitable, since the role size is not part of the search.
    
    
    \item Crossover is just a bigger mutation
    \item The optimization of combining roles with similar roles or similar permissions is not necessarily optimal?
    \item A smarter mutation, which does not allow "invalid" role models is...
    
    
    \item Individuals might have the same Fitness and even re-created the original UP-Matrix, but it does not mean that the role model is the same; Additional attributes, like user attributes, could help by e.g. introducing an Interpretability measure to distinguish the role models
    
	\item Multiobjective EA shows ...
	\item Multiobjective EA can ...
	\item Noise in the data can ...
	\item Constraint handling can be easily introduced, but...
	\item Rolemodel is hard to evaluate as a whole by a human
    \item The alternative approach is not developed far enough to see any advantage ...
\end{itemize}