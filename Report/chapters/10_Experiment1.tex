\section{Experiment 1 - Single Objectives}
\label{sec:exp1}
For the first experiment single objective fitness functions are tested on the Evo-RoleMiner. The objectives tested are:
\begin{itemize}
	\item Confidentiality Violations (Experiment 1a)
	\item Availability Violations (Experiment 1b)
	\item Role Count (Experiment 1c)
	\item User-Role-Assignment Count (Experiment 1d)
	\item Role-Permission-Assignment (Experiment 1e)
	\item Average Role Interpretability (Experiment 1f)
\end{itemize}
These are the objective measures introduces in section \ref{sec:objectiveMeasure}. All single objective fitness functions are formulated as minimization function except the fitness function with objective "Average Role Interpretability", which is a maximization fitness function.
The experiments are executed in order to validate the objective measures and to analyse the single objectives in relation to each other.

A basic setup is chosen seen in Table \ref{tab:exp1_setup}. The experiments are based on the Dataset1.

\begin{table}[H]
    \centering
    \begin{adjustbox}{width=0.5\textwidth}
	    \begin{tabular}{|l|l|}
	        \hline
	        \rowcolor{gray!25} 
	        \textbf{Parameter}              & \textbf{Value}    \\ \hline
	        Generations                     & 100              \\ \hline
	        Population                      & 100              \\ \hline
	        CXPB                            & 0.25              \\ \hline
	        MUTPB                           & 0.25              \\ \hline
	        MUTPB-Type1: Add role           & 0.25              \\ \hline
	        MUTPB-Type2: Add User           & 0.25              \\ \hline
	        MUTPB-Type3: Add Permission     & 0.25              \\ \hline
	        MUTPB-Type4: Remove Role        & 0.25              \\ \hline
	        MUTPB-Type5: Remove User        & 0.25              \\ \hline
	        MUTPB-Type6: Remove Permission  & 0.25              \\ \hline
	        Tournament size                 & 2                 \\ \hline
	        Local optimization              & True              \\ \hline
	    \end{tabular}
	\end{adjustbox}
    \caption{EXPERIMENT 1 setup}
    \label{tab:exp1_setup}
\end{table}

\begin{figure}[H]
    \centering
    \includegraphics[scale=0.33, trim=4cm 2cm 4cm 3cm, clip=true]{exp1_conf}
    \caption{EXPERIMENT 1a: Results of Evo-RoleMiner with Fitness function $F=G_{conf}$ on Dataset1 with setup in table \ref{tab:exp1_setup}. From upper left to lower right: Confidentiality Violations, Availability Violations, Role Count, User-Role Assignments, Role-Permission Assignments, Interpretability.}
    \label{fig:exp1_conf}
\end{figure}

For the single objective "Confidentiality Violations" it is expected that the individuals (role models) violate confidentiality compared to the original access configuration $UPA$ less over time. The results in Figure \ref{fig:exp1_conf} are confirming this and also show that individuals have less roles over time. This can be explained by that a lower count of roles probably result in less user-role- and role-permission assignments, which on the other hand lowers the probability of violating confidentiality. At the same time less user-role- and role-permission assignments let the probability for availability violations rise, which can be also seen in Figure \ref{fig:exp1_conf}.

\begin{figure}[H]
    \centering
    \includegraphics[scale=0.33, trim=4cm 2cm 4cm 3cm, clip=true]{exp1_accs}
    \caption{EXPERIMENT 1b: Results of Evo-RoleMiner with Fitness function $F=G_{accs}$ on Dataset1 with setup in table \ref{tab:exp1_setup}. From upper left to lower right: Confidentiality Violations, Availability Violations, Role Count, User-Role Assignments, Role-Permission Assignments, Interpretability.}
    \label{fig:exp1_accs}
\end{figure}

An opposite impact can be seen when single objective "Availability Violations" is used as fitness function (see Figure \ref{fig:exp1_accs}). In the first generations the availability violations quickly achieve the goal of zero. During the same generations the role count, user-role- and role-permission assignments are rising. It is more likely that every user gets his permission (given in the original UPA-Matrix), if the amount of roles, user-role- and role-permission assignments are higher.

Hence, the objectives of minimizing confidentiality violations and minimizing availability violations can be conflicting. Other results of the first phase can be seen in the Appendix \ref{sec:A_Exp1}.