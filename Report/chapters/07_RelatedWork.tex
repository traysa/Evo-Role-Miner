\newpage
\section{Related Work}

\hl{Contents:}
\begin{itemize}
\item \hl{A survey of the literature (journals, conferences, book chapters) on the areas that are relevant to your research question. One section per area.
The chapter should conclude with a summary of the previous research results that you want to develop further or challenge. The summary could be presented in a model, a list of issues, etc. Each issue could be a chapter in the presentation of results. They should definitely be discussed in the discussion / conclusion of the thesis.}
\item \hl{The Literature Review provides the necessary background information to familiarize the reader with prior research and relevant theory.  Three general types of literature reviews exist:  the broad scan, the focused review, and the comprehensive critique.}
\end{itemize}

\begin{itemize}
\item \hl{More than a literature review}
\item \hl{Organize related work - impose structure}
\item \hl{Be clear as to how previous work being described relates to your own.}
\item \hl{The reader should not be left wondering why you've described something!!}
\item \hl{Critique the existing work - Where is it strong where is it weak? What are the unreasonable/undesirable assumptions?}
\item \hl{Identify opportunities for more research (i.e., your thesis) Are there unaddressed, or more important related topics?}
\item \hl{After reading this chapter, one should understand the motivation for and importance of your thesis}
\item \hl{You should clearly and precisely define all of the key concepts dealt with in the rest of the thesis, and teach the reader what s/he needs to know to understand the rest of the thesis.}
 
\end{itemize}

% Matrix Factorization
%First mentioned: Yannakakis 1990
% In machine learning: Lee, Seung 1999
%https://www.youtube.com/watch?v=kSfwY68gQ9I

\subsection{Role Mining with Data Mining}
Role Mining has been first coined in \cite{Kuhlmann}. In the following years several researchers have analyzed Role Mining further and defined several Role Mining Problems, Quality Measures, Cleaning Techniques and Algorithms.
\subsection{Role Mining of "Meaningful Roles"}
In the recent years several researchers are investigating the problem finding "meaningful" roles, since the classic Role Mining approaches outputs RBAC models, which are often not accepted in practice.
\subsection{Role Mining with Bio-inspired Techniques}
In \cite{Igor} the basic RMP is tackled with an evolutionary algorithm. In a first version of the approach the authors use a specific representation of the phenotypes. They change this representation in an improved approach. In this thesis the suggested improved approach is used as starting point in my approach to deal with the RMP. In a recent paper \cite{Igor2} the authors concentrate on their first approach again.\\
In \cite{DuChang} the authors use two different Artificial Intelligence (AI) approaches to do Role Mining. In one approach they use a genetic algorithm. The second approach uses an Ant Colony approach. The result shows that...\\
In \cite{paper} an evolutionary approach for solving the policy generation problem is introduced. 


