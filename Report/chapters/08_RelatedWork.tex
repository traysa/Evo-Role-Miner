\newpage
\chapter{Related Work}
\label{sec:relatedWork}

\iffalse
\begin{itemize}
\item \hl{A survey of the literature (journals, conferences, book chapters) on the areas that are relevant to your research question. One section per area.
The chapter should conclude with a summary of the previous research results that you want to develop further or challenge. The summary could be presented in a model, a list of issues, etc. Each issue could be a chapter in the presentation of results. They should definitely be discussed in the discussion / conclusion of the thesis.}
\item \hl{The Literature Review provides the necessary background information to familiarize the reader with prior research and relevant theory.  Three general types of literature reviews exist:  the broad scan, the focused review, and the comprehensive critique.}
\end{itemize}

\begin{itemize}
\item \hl{More than a literature review}
\item \hl{Organize related work - impose structure}
\item \hl{Be clear as to how previous work being described relates to your own.}
\item \hl{The reader should not be left wondering why you've described something!!}
\item \hl{Critique the existing work - Where is it strong where is it weak? What are the unreasonable/undesirable assumptions?}
\item \hl{Identify opportunities for more research (i.e., your thesis) Are there unaddressed, or more important related topics?}
\item \hl{After reading this chapter, one should understand the motivation for and importance of your thesis}
\item \hl{You should clearly and precisely define all of the key concepts dealt with in the rest of the thesis, and teach the reader what s/he needs to know to understand the rest of the thesis.}
\end{itemize}
\fi

% Matrix Factorization
%First mentioned: Yannakakis 1990
% In machine learning: Lee, Seung 1999
%https://www.youtube.com/watch?v=kSfwY68gQ9I

Role Mining has been first coined in Kuhlmann et al.\cite{Kuhlmann}. In the following years several researchers have analyzed Role Mining further and defined several Role Mining problems, quality measures, cleaning techniques and algorithms. In the following the related work to this thesis is introduced.

\section{Role Mining of "Meaningful Roles"}
\label{sec:relatedWork2}
In the recent years several researchers are investigating the problem finding "meaningful" roles, since the classic Role Mining approaches output RBAC models, which are often not accepted in practice. In Frank et al.\cite{Frank} a probabilistic role mining approach is combined with business information to a hybrid role mining algorithm. Where as in Xu \& Stoller\cite{Xu} a role mining algorithm based on bottom-up data is creating candidate roles, which then get post-processed with business information to determine meaningful roles.

There are two algorithms introduced by Xu \& Stoller\cite{Xu}, which both consists of three phases. Both algorithms start with the same phase of candidate role generation. In the first algorithm, the "Elimination Algorithm", the generated candidate roles build the initial role model, which then gets reduced by role elimination. At last a role restoration phase is executed. In the second algorithm, the "Selection Algorithm", on the other hand the initial role model is empty and is then filled by role selection from the candidate roles. The algorithm ends with a role pruning phase.

For the candidate role generation the authors extend the role mining algorithm in Vaidya et al. (CompleteMiner)\cite{Vaidya:2006:RMR:1180405.1180424}, which is mining roles by subset enumeration, with inheritance relation of roles. The output is therefore a candidate role hierarchy. But the authors mention that also other approaches can be used to determine a candidate role hierarchy, like e.g. frequent pattern tree (FP-tree\cite{Han}) approach.

In the role elimination phase of the elimination algorithm roles get repeatedly removed from the initial role model as long the removal causes no violations according to the given access policy configuration and improves the quality of the role model. The role with the worst role quality gets removed first. With every removal the role hierarchy and the roles get adjusted, such that no violations occur. In the third phase of the elimination algorithm the previously removed roles get restored in the same order if the role model quality gets improved. Also here the role hierarchy and roles get adjusted accordingly.

The role model quality gets measured by the Weighted Structural Complexity (WSC) and role model interpretability. The WSC is a sum of the count of roles, user-role assignments, role-permission assignments and role-role assignments. Each summand has a weight, which can be adjusted according to the importance. The interpretability of a role model measures how well the users in the roles can be expressed by user attribute expressions. User attribute expressions are basically IF-THEN-rules, for example "$Department=Sales \wedge Location=Denmark$". The count of users, who do not fulfill this user attribute expression, is called attribute mismatch. For each role and its users the minimum attribute mismatch is chosen among all attribute mismatches of all possible combinations of user attribute expressions. The interpretability of a role model is the sum of the minimum attribute mismatch of each role in the role model.

For the role quality the authors suggest three measures: Clustered Size, Attribute fitness and Redundancy. The clustered size of a role measures the number of access policy configurations covered by the role in relation to the number of users of the role. The attribute fitness of a role is based on the minimum attribute mismatch of the role. The redundancy of a role is measured by how many other roles cover the same fraction of the access policy configurations.

The Selection Algorithm works in the opposite way as the elimination based algorithm: Candidate roles are repeatedly added to the role model by choosing the role with the best role quality first. This is continued until the role model does not violate the given access policy configuration. In the third phase the roles in the role model get checked in the reverse order that they have been added. The check determines if a role can be removed, such that the given access policy configuration is not violated, and if the removal would improve the role model quality. If the check is positive the role gets removed from the rolemodel.

The experiments are executed on public available datasets, which initially only provide access policy configuration information, but have no user attribute information. The authors generate synthetic user attribute information for the public available datasets by advanced computation.

The authors compare both algorithms and show that the elimination algorithm performs better than the selection algorithm. Furthermore the elimination algorithm is compared with other role mining algorithm, i.e. HierarchicalMiner\cite{Molloy:2010}. The results show that equal or better results than previously proposed algorithms can be achieved according to WSC and Interpretability. For future work the authors suggest to consider other interpretability measures, which consider heterogeneity of users in different roles as well as homogeneity of users in the same role.

In Du \& Chang\cite{DuChang} the authors exchange the Elimination Algorithm with a GA-based algorithm and the Selection algorithm with an Ant-Colony-Optimization (ACO)-based algorithm. The authors compare both algorithms and show that the GA-based approach produces better results than the ACO-based approach in consideration of the objectives. Furthermore the results are compared with the results in Xu \& Stoller\cite{Xu} and it is shown that the proposed algorithms achieve better performance in consideration of the objectives.

The created EA in this thesis is tested on the same datasets used by the papers. In contrast to the introduced papers, the approach in this thesis is to involve the interpretability measure when generating a role model from scratch, instead of optimizing a pre-mined set of candidate roles. Furthermore a different interpretability measure is introduced as used in Xu \& Stoller\cite{Xu} and Du \& Chang\cite{DuChang}, which is not only counting the attribute mismatch, but uses other evaluation measures from data mining.
The provided synthetic user attribute information in Xu \& Stoller\cite{Xu} is restricted by the given public datasets. To be able to construct input data, which contains access policy configurations and user attribute information, with various size and complexity a data generator is created. With an own data generator synthetic datasets in different sizes can be generated to test this thesis' approach.

\section{Role Mining with Bio-inspired Techniques}
\label{sec:relatedWork3}\
The work of Saenko \& Kotenko \cite{Igor} \cite{saenko2012design} is the only known approach of solving the Basic RMP and the Min-Edge RMP with a genetic algorithm (GA), where a role model should be generated from scratch.

For the evaluation function of the GA the authors combine several objectives in one single objective maximization function, where the fitness values for each objective are weighted. The objectives for the Basic RMP are the minimization of roles, confidentiality violations (overentitlements) and availability violations (underentitlements). A confidentiality violation expresses a situation where a user would get an access right due to the generated role model, which he or she did not had in the current access control configuration. An availability violation on the other hand expresses if a user gets an access right less than he or she did had in the current access control configuration. For the Min-Edge RMP\cite{4497438} the objective of minimizing roles is exchanged with minimizing the number of user-role- and role-permission assignments.

In a first version of the approach the authors choose a representation of role models as individuals consisting of three chromosomes: A chromosome $X$, which represents the UA-Matrix, a chromosome $Y$, which represents the PA-Matrix, and a control-chromosome $Z$, which controls if a role is active or passive. With the control-chromosome they influence how many roles a role model has and make it possible to vary the amount with variation operators of the GA. The authors suggest a crossover function in three phases, which is particularly designed for the suggested multi-chromosomal representation. The drawback of this approach are unnecessary passive genes. The chromosomes X and Y, respectively the UA- and PA-Matrix, contain roles, which are passive (controlled by chromosome Z).

Due to this drawback the authors suggest an improved representation where they combine the X and Y chromosome into one variable-length chromosome and remove the control-chromosome Z. A role model is now representation as chromosome where a gene represents a role. A gene (role) consists of a user-list ($L_X$) and a permission-list ($L_Y$). Due to the change of representation the crossover method is changed accordingly.

The first approach is only tested on randomly generated data with three different dimensions\cite{Igor}. The evaluation is only on the number of generations needed to obtain a solution which meet the suggested evaluation function. The performance of the first and the improved GA are compared with the result that the improved GA has a better performance in all dimensions of the synthetic input UPA, especially in greater dimensions\cite{saenko2012design}. No further evaluations on the resulting role models in the suggested approaches are shown. In the recent paper \cite{Kotenko:2015} the authors focusing on the multi-chromosomal approach again.

In this thesis' approach the suggested improved version of the EA of Saenko \& Kotenko\cite{saenko2012design} is used as starting point. The same representation of individuals, fitness functions and crossover operation is re-implemented. Other parts of the EA, like mutation operators or selection strategy, are not mentioned in the paper and are therefore guided by standards or intuitive ideas. Furthermore a MOEA is used in this thesis, which does not require to combine several objectives to one fitness function by scalarization. The re-implemented EA and the MOEA are tested on datasets, which are commonly used in the role mining research, and analysis the results on several objectives.