\newpage
\section{Introduction}

\iffalse
\hl{Content:}
\begin{itemize}
    \item \hl{On the first page, you should present}
        \subitem \hl{The area of research (e.g. implementation of information systems)}
        \subitem \hl{The most relevant previous findings in this area}
        \subitem \hl{Your research problem and why this is worthwhile studying}
        \subitem \hl{The objective of the thesis: how far you hope to advance knowledge in the field}
    \item \hl{Target group}
        \subitem \hl{To whom are you writing, what do you assume that the reader know? A normal target groupd would be new Master students.}
    \item \hl{Personal motivation}
        \subitem \hl{Why did you choose this topic?}
    \item \hl{Research method in brief}
        \subitem \hl{How will you find out?}
    \item \hl{Structure of the report}
        \subitem \hl{A paragraph about each chapter. What is the main contribution of the chapter? How do they relate?}
\end{itemize}

\begin{itemize}
    \item \hl{The primary purpose of the introduction is to provide the reader with an overview of the study itself and the influencing factors in its development. This section should briefly introduce the setting and methods used in the study and present the study purpose and hypothesis. The first two to three paragraphs of this section should focus on summarizing the nature of the thesis, including the writer's motivation for choosing the topic. Next, the writer should discuss the significance of the topic in relation to the setting or the framework in which the study occurs. In addition, the writer should include an acknowledgement of the prior research or information upon which the study is based. The scope of the study should be presented, along with a general description of what the reader can expect in the remainder of the document. Finally, the introduction should end with a brief discussion of what the writer anticipated will be the value of the research project. }
\end{itemize}

\begin{itemize}
    \item \hl{Thesis Statement (one or two sentences)}
        \subitem \hl{What is your thesis about and what have you done?}
        \subitem \hl{If you have a hypothesis what is it?}
        \subitem \hl{How will you test (prove/disprove) your hypothesis?}
    \item \hl{Motivation}
        \subitem \hl{Why is this problem you've worked on important}
    \item \hl{Goals / Objectives}
        \subitem \hl{What are you trying to do and why?}
        \subitem \hl{How will you or the reader know if or when you've met your objectives?}
    \item \hl{**** Contributions *****}
        \subitem \hl{What is new, different, better, significant?}
        \subitem \hl{Why is the world a better place because of what you've done?}
        \subitem \hl{What have you contributed to the field of research?}
        \subitem \hl{What is now known/possible/better because of your thesis?}
    \item \hl{Outline of the thesis (optional)}
\end{itemize}
\fi

\iffalse
The main purpose of \gls{IAM} is to control access to resources by using security policies. \glspl{ACL} are maintaining security policies as direct assignments from access rights (permissions) to \glspl{Identity} (users, computers or other principals). In order to lower the complexity and cost of access control administration, access control models are introduced. The most used access control model in the recent years in enterprise identity management systems is \gls{RBAC}. \cite{Kunz} \gls{RBAC} security policies are grouping permissions into roles, which then are assigned to users.\\

\iffalse
Role-based access control (RBAC) is the most used access control model in enterprise identity management systems nowadays \cite{Kunz} due to the advantage to lower the complexity and cost of access control administration. But also Attribute-based access control (ABAC) is getting more popular due to its higher flexibility. Both access control models have their advantages and disadvantages. \cite{Hu13guideto} \cite{Coyne:2013}\\
\fi 

\iffalse
Access Control Lists (ACL) are maintaining security policies, which can be represented as tuples <U,P,UP>. Here, U is representing a set of users (identities), P denotes a set of permissions (access rights) and UP ⊆ U × P denotes a user-permission assignment.
In RBAC security policies are defined by tuples <U,P,R,UR,RP,RR>. \cite{DuChang}
\fi

One major task of \gls{RBAC} is the role engineering, the creation of a Role Model. There are two approaches to build a role model: Top-Down and Bottom-Up. While in the Top-Down approach the role model is created by defining enterprise roles out of business processes, like e.g. the job descriptions, the Bottom-Up approach tries to create reasonable enterprise roles by taking current assignments of users to permissions into account. The Top-Down approach has the disadvantage that the designed roles do not contain all current access right assignments, which could lead to problems, when the model is applied. Furthermore the top-down approach is a long process with high costs. In the Bottom-Up approach data mining techniques are applied on current assignments of users to permissions in order to get results in reasonable time. But often these results do not have any business meaning, so that they do not get adopted by managers and system administrators, which are responsible for the correct assignment of these roles. Several workshops are needed to transform the mined role model into a realistic role model, which is accepted by the business. A hybrid model is combining data mining techniques with business knowledge to exploit the advantages of both approaches. \cite{Frank} \cite{Xu} \cite{Coyne:2011}

The desired goal in role mining is finding an \gls{RBAC} model, which can lower the complexity and cost of access control administration the most. The measuring of the current quality of a role model and selecting criteria for its optimization is still unsolved. There are several quality measures for an \gls{RBAC} model, which can be dependent on the enterprises policy. The quality measures of an \gls{RBAC} model can be rely on the overall \gls{RBAC} state or on single roles within the \gls{RBAC} model. The most addressed quality measures in existing role mining algorithms are a.o. achieving completeness, minimize the number of roles, decrease role set similarity, increasing role coverage, fulfill role constraints, Minimize Users/Permissions per Role and Minimize/Maximize Roles per User/Permission. \cite{Kunz} 

\iffalse
The desired goal in role mining is finding the minimum number of meaningful roles which are comprehending all assignments of users to permissions in regard of the least privileged access principle. The least privileged access principle means giving a user account only those privileges which are essential to that user's work. Meaningful roles are roles which are accepted by the business.\\
\fi

There are several challenges in role mining. When taking business information into account, relevant and usable business information has to be identified and applied within the mining process. Relevant business information could be e.g. the hierarchy of \glspl{OrgU} or location of an user. The quality of the data has also a big influence on the result. This is not only influenced by noise removal of the data set, but also which data set is used.
\iffalse
In order to meet the least privileged access principle, a data set, which describes the actual user activity and usage of permissions is preferable. These information can come from system logs, which truly records the permission-usage and permission-update information. \cite{Molloy}\\
\fi

Additionally the mined role model should take constraints into account, such as \gls{SoD} and exceptions. \cite{Lu}\\

Another challenge in \gls{RBAC} is the maintaining of the role model. Permissions, Users and business information are evolving over time and the role model might become suboptimal over time. There is a practical need for periodic quality assessment of the resulting role models. \cite{Kunz} A complete re-modelling would not be a feasible solution, since the system administrators and managers would not accept if the role model is regularly restructured. A solution which slowly evolves the role model with the occurring changes has to be designed.


When users with domain knowledge interact within the role mining process, an appropriate visualisation of role mining results has to be developed. Also how to incorporate background domain knowledge to evaluate the results or to guide the search towards a result has to be considered. \cite{Han}\\

\iffalse
Currently researchers are looking into using artificial intelligence (AI) techniques to design role mining algorithms. \cite{DuChang}
\fi

\fi