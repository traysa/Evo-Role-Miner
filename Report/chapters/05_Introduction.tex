\newpage
\chapter{Introduction}
\iffalse
    \hl{Content:}
    \begin{itemize}
        \item \hl{On the first page, you should present}
            \subitem \hl{The area of research (e.g. implementation of information systems)}
            \subitem \hl{The most relevant previous findings in this area}
            \subitem \hl{Your research problem and why this is worthwhile studying}
            \subitem \hl{The objective of the thesis: how far you hope to advance knowledge in the field}
        \item \hl{Target group}
            \subitem \hl{To whom are you writing, what do you assume that the reader know? A normal target groupd would be new Master students.}
        \item \hl{Personal motivation}
            \subitem \hl{Why did you choose this topic?}
        \item \hl{Research method in brief}
            \subitem \hl{How will you find out?}
        \item \hl{Structure of the report}
            \subitem \hl{A paragraph about each chapter. What is the main contribution of the chapter? How do they relate?}
    \end{itemize}
    
    \begin{itemize}
        \item \hl{The primary purpose of the introduction is to provide the reader with an overview of the study itself and the influencing factors in its development. This section should briefly introduce the setting and methods used in the study and present the study purpose and hypothesis. The first two to three paragraphs of this section should focus on summarizing the nature of the thesis, including the writer's motivation for choosing the topic. Next, the writer should discuss the significance of the topic in relation to the setting or the framework in which the study occurs. In addition, the writer should include an acknowledgement of the prior research or information upon which the study is based. The scope of the study should be presented, along with a general description of what the reader can expect in the remainder of the document. Finally, the introduction should end with a brief discussion of what the writer anticipated will be the value of the research project. }
    \end{itemize}
    
    \begin{itemize}
        \item \hl{Thesis Statement (one or two sentences)}
            \subitem \hl{What is your thesis about and what have you done?}
            \subitem \hl{If you have a hypothesis what is it?}
            \subitem \hl{How will you test (prove/disprove) your hypothesis?}
        \item \hl{Motivation}
            \subitem \hl{Why is this problem you've worked on important}
        \item \hl{Goals / Objectives}
            \subitem \hl{What are you trying to do and why?}
            \subitem \hl{How will you or the reader know if or when you've met your objectives?}
        \item \hl{**** Contributions *****}
            \subitem \hl{What is new, different, better, significant?}
            \subitem \hl{Why is the world a better place because of what you've done?}
            \subitem \hl{What have you contributed to the field of research?}
            \subitem \hl{What is now known/possible/better because of your thesis?}
        \item \hl{Outline of the thesis (optional)}
    \end{itemize}
\fi
Access control policies are maintaining security policies as direct assignments from access rights (permissions) to identities (users, computers or other principals). In order to lower the complexity and cost of access control administration, access control models were introduced. The most used access control model in the recent years in enterprise identity management systems is the Role-based Access Control (RBAC)\cite{Kunz}. RBAC security policies are grouping permissions into roles, which then are assigned to identities. Permissions can be contained in several roles and users can be assigned to several roles.

One major task of RBAC is the role engineering, the creation of a role model, which determines which permissions are grouped together to roles and to whom are they assigned. Furthermore how many roles and assignments are needed, to cover the current access control policies. In enterprises the amounts of users and permissions can be enormous, which made the usage of data mining or artificial intelligence techniques popular. The aim is to discover a good role model, which covers the current access control policies and effectively lowers the complexity and cost of access control administration. These process of obtaining role models with the help of data mining or artificial intelligence techniques is called role mining.

In this thesis the use of an evolutionary computation approach for role mining is investigated.

\iffalse
There are two approaches to build a role model: Top-Down and Bottom-Up. While in the Top-Down approach the role model is created by defining enterprise roles out of business processes, like e.g. the job descriptions, the Bottom-Up approach tries to create reasonable enterprise roles by taking current assignments of users to permissions into account. The Top-Down approach has the disadvantage that the designed roles do not contain all current access right assignments, which could lead to problems, when the model is applied in production. Furthermore the Top-Down approach is a long process with high costs. In the Bottom-Up approach data mining or artificial intelligence techniques are applied on current assignments of users to permissions in order to get results in reasonable time. But often these results do not have any business meaning, so that they do not get adopted by managers and system administrators, which are responsible for the correct assignment of these roles. Several workshops are needed to transform the mined role model into a realistic role model, which is accepted by the business. A hybrid model is combining data mining techniques with business knowledge to exploit the advantages of both approaches.\cite{Frank} \cite{Xu} \cite{Coyne:2011}

The desired goal in role mining (Bottom-Up approach) is finding an RBAC model, which can lower the complexity and cost of access control administration the most. The measuring of the current quality of a role model and selecting criteria for its optimization is still unsolved. There are several quality measures for an RBAC model, which can be dependent on the enterprises policy. The quality measures of an RBAC model can rely on the overall RBAC state or on single roles within the RBAC model. The most addressed quality measures in existing role mining algorithms are, among others, achieving completeness, minimize the number of roles, decrease role set similarity, increasing role coverage, fulfill role constraints, Minimize Users/Permissions per Role and Minimize/Maximize Roles per User/Permission.\cite{Kunz}

There are several challenges in role mining. When taking business information into account, relevant and usable business information has to be identified and applied within the mining process. Relevant business information could be e.g. the hierarchy of organizational unit or the respective costcenter of an user. The quality of the data has also a big influence on the result. This is not only influenced by noise removal of the data set, but also which data set is used. Additionally the mined role model should take constraints into account, such as Separation of duties (SoD) and exceptions\cite{Lu}.

Another challenge in RBAC is the maintaining of the role model. Permissions, Users and business information are evolving over time and the role model might become suboptimal over time. There is a practical need for periodic quality assessment of the resulting role models\cite{Kunz}. A complete re-modelling would not be a feasible solution, since the system administrators and managers would not accept if the role model is regularly restructured. A solution which slowly evolves the role model with the occurring changes has to be designed.

When users with domain knowledge interact within the role mining process, an appropriate visualization of role mining results has to be developed. Also how to incorporate background domain knowledge to evaluate the results or to guide the search towards a result has to be considered\cite{Han}.
\fi


\section{Motivation}
Role mining is an optimization problem, which seeks for an optimal role model. Selecting the criteria for its optimization is still unsolved. There are several quality measures for an RBAC model, which can be dependent on the enterprises policy. Clearly the role mining optimization is following several objectives, for example minimizing the number of roles and minimizing the number of confidentiality and availability violations (over- and underentitlements), which can be in conflict with each other.

With multiobjective evolutionary algorithms pareto-optimal solutions, a set of optimal trade-off solutions, can be found in a single run instead of several runs, which is necessary for some of the conventional stochastic processes, like simulated annealing and tabu search\cite{abraham2005evolutionary}. Previous research in the role mining field are using scalarisation, where several objectives are weighted and summarized in one optimization function, like the Weighted Structural Complexity (WSC) in Molloy et al.\cite{Molloy} or the fitness functions in Saenko \& Kotenko\cite{saenko2012design}. The trade-offs between the objectives are not apparent in these approaches.

There are constraining policies existing, such as Separation of duties (SoD), which should prevent that certain permissions are assigned to a user at the same time to avoid fraud. Role mining algorithm might not take this into account and the resulting role model is corrected in a post process by domain experts.

Evolutionary algorithms can include constraint handling by for example penalize role model solutions, which do not comply to the policies. 

The resulting role models of role mining often do not have any business meaning, so that they do not get adopted by managers and system administrators, which are responsible for the correct assignment of these roles. Several workshops with domain experts are needed to transform the mined role model into a realistic role model, which is accepted by the business.

One approach is to include business information into the role mining algorithm. But first relevant and usable business information has to be identified and collected. Furthermore a measure for the "meaningfulness" of roles has to be identified.

Another approach could be to include human expert knowledge to a role mining algorithm. Interactive evolutionary computation (IEC)\cite{949485} could accomplish human feedback to a role mining algorithm based on evolutionary computation to navigate the algorithm to an optimal result, accepted by the business.

Another challenge in RBAC is the maintaining of the role model. Permissions, users and business information are evolving over time and the initially created role model might become suboptimal over time. There is a practical need for periodic quality assessment of the resulting role models\cite{Kunz}. A complete re-modelling would not be a feasible solution, since the system administrators and managers would not accept if the role model is regularly restructured. A solution which slowly evolves the role model with the occurring changes has to be designed.

Evolutionary algorithms can adapt solutions to changing circumstances\cite{Fogel:1997}. A current role model state can be used as starting population and the evolutionary algorithm will optimize towards the formulated objectives.

There are several arguments, which are motivating to use an evolutionary computation approach for role mining, but so far this approach has not been researched very widely. Only few research (Saenko \& Kotenko \cite{Igor}\cite{saenko2012design}) could be found, which started to investigate to mine a role model with an evolutionary algorithm. 

In this thesis the suggested evolutionary algorithm by Saenko \& Kotenko is implemented as a starting point, where the not mentioned parts of the algorithm are intuitively filled. The algorithm is tested on synthetic and real datasets commonly used in role mining research. Furthermore multi-objective evolutionary algorithms are implemented tested. Challenges of the approach are pointed out. Another topic investigated is to include business information data into the role mining algorithm.

\section{Thesis question}
How can a role model be mined from scratch with an evolutionary computation approach \hl{(and how adequate are the results)}? Can business information data improve the evolutionary computation approach?

\section{Roadmap}
The remainder of this thesis report is organized as follows. The thesis starts with an introduction to the domain of RBAC and Role Mining (Chapter \ref{sec:domain}). Afterwards an introduction to evolutionary algorithms is given, where the basic concepts are explained and a description of algorithms is given (Chapter \ref{sec:EA}). With the background information to the domain and evolutionary algorithms in mind the related work is described in Chapter \ref{sec:relatedWork}. In Chapter \ref{sec:approach} the approach and implementation is described and decisions are discussed. The experiments and according results can be read in Chapter \ref{sec:experiments}. A discussion on the results and the approach is given in Chapter \ref{sec:discussion}. Chapter \ref{sec:futureWork} gives an outlook for future work. At last a conclusion is given in Chapter \ref{sec:conclusion}, where the work is summarized and the most important findings are highlighted.