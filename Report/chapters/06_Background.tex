\newpage
\section{Background: Domain}
In this section a basic introduction to the domain is given. It is important that the motivation and concepts of the domain are known in order to scope the problem definition and for guidance in the implementation of the approach.

    \subsection{Access Control Models}
    Access control ensures through policy definitions and enforcement that users\footnote{For simplicity the term "User" is used throughout the thesis although the term "subject" would be more general, since they do not only include a user, but also service accounts or any other subject} can only access resources (e.g. files, applications, networks) they are authorized for. The policy definitions describe which user is authorized for which permission. A permission describes which action (e.g. read, write, execute) can be done on a resource.\\
    There are business, security and regulatory drivers for managing access control policies \cite{o20102010}. The interest of the business is to lowering the costs for managing the permissions for employees and quickly equip employees with necessary permissions such that they can efficiently full-fill their tasks. The security driver is to ensure information security, integrity, and availability to prevent accidental and intentional security breaches. But also to be compliant with regularities, such as the Data Protection Directive in Europe (Directive 95/46/EC)\footnote{\url{http://eur-lex.europa.eu/LexUriServ/LexUriServ.do?uri=CELEX:31995L0046:en:HTML; 07-10-2015}}, Basel II\footnote{\url{https://en.wikipedia.org/wiki/Basel_II; 07-10-2015}} or company regulations, have an impact on the access control policies.\\
    The National Institute of Standards and Technology (NIST) \cite{Hu13guideto} describes various logical access control models, which provide a policy framework that specifies how permissions are managed and who, under what circumstances, is entitled to which permission and enforce these access control decision. Some of these logical access control models are briefly described in the following to enable the reader to differentiate between the concepts.
    \iffalse There is no consensus on the terms access control models, mechanisms and techniques. In this thesis an access control policy model describes a policy framework that specifies how permissions are managed and who, under what circumstances, is entitled to which permission on a high-level. There are different Access Control Policy Models which come with different advantages and disadvantages. Some of these models are briefly described in the following to enable the reader to differentiate between the concepts. \fi
    \begin{itemize}
        \iffalse \item \textbf{Discretionary Access Control (DAC)}\\\fi
        \item \textbf{Identity-based Access Control (IBAC)}\\
        In IBAC a user gets a certain access to a resource by being assigned directly to a permission, which is connected to a resource. On a low-level Access Control Lists (ACLs) are implementations of IBAC. While IBAC may be manageable in companies with a small amount of users and permissions, the maintenance can be quickly overwhelming in consideration of satisfying all drivers of access control policies (see above).
        \item \textbf{Role-based Access Control (RBAC)}\\
        In RBAC permissions are bundled to roles, which then are assigned to users. A user gets a certain access to a resource by being assigned to a role, which contains the corresponding permission to that access. In other words users inherit all permissions of the roles they are assigned to. The motivation of this extra abstraction layer of a role is to easier maintain the access control of users.
        Even more abstraction can be introduced by role hierarchies (see next section).\iffalse , where a role inherits all permissions of its parent-role.\fi The RBAC model, which contains the roles, the user assignments to roles, the permission assignments to roles and the role-hierarchy, needs to be defined before the access control mechanism can enforce the access control. The RBAC model can be an ease of administration in comparison to direct assignments in IBAC. The degree of the benefit is dependent on the RBAC model.
        \item \textbf{Attribute-based Access Control (ABAC)}\\
        In \cite{Hu13guideto} the authors try to guide to a standard definition of ABAC, since there seems to be no consensus. The basic concept of ABAC is to introduce an additional abstraction layer in form of ABAC policies, which are basically complex boolean rule sets that can evaluate different kinds of attributes. These attributes could be user information (e.g. department, job title), resource information (e.g. threat level) or environment information (e.g. current time, current location). A user gets a certain access to a resource if the rule set is satisfiable. The rule sets need to be defined before the access control mechanism can enforce the access control. Compared to RBAC ABAC seems more flexible, but it also comes with challenges regarding risk and auditing\cite{Coyne:2013}.
    \end{itemize}
    The scope of this thesis is to research an approach which outputs an RBAC model. Some ideas of the ABAC model are exploited to some extent in the implementation of the thesis approach.\\
    
    \subsection{Functional capabilities of RBAC}
    The RBAC model is often used in bigger organizations\cite{o20102010} and is leveraged by many Identity- and Access Management (IAM) systems. There are different functional capabilities of RBAC models. The NIST RBAC model \cite{sandhu2000nist} distinguished between four levels of RBAC models: Flat, Hierarchical, Constrained and Symmetric RBAC. Each level introduces an additional functional capability to the previous level. Some of the functional capabilities of an RBAC model are introduced in the following.
    \begin{itemize}
        \item \textbf{Basic RBAC Capability}\\
        In the basic RBAC model roles are bundles of permissions and users are assigned to these roles in order to get the according permissions. A user can be assigned to several roles and roles can be assigned to several users. The same many-to-many relation exists between roles and permissions. Therefore a user get can get the same permission several times by different roles.
        \item \textbf{Role Hierarchy}\\
        In a role hierarchy roles can be assigned to roles and inherit their permissions to the roles they have been assigned to. The role hierarchy can be restricted as a tree or inverted tree, where a role inherits only the permissions of its child-roles, or a partially ordered set, where a role inherits the permissions from any other role, which is assigned to it. It is also possible to limit inheritance in role hierarchies to control the power of impact of roles.
        \item \textbf{Separation-of-Duties (SoD)}\\
        Separation-of-Duties, also known as Segregation-of-duties, is a security principle for preventing fraud and errors by splitting tasks and associated permissions among multiple users. For example a user who has the permission to request the purchase of goods or services should not have the permission of approving the purchase. Roles in an RBAC model should therefore not have permissions bundled, which violate each other due to SoD. But also user assignments to multiple roles should not violate the SoD requirements.
    \end{itemize}
    In this thesis the main focus will be on the basic RBAC Capability and SoD Capability.\\
    
    \subsection{Role Model}
    A role model describes the roles, user-role assignments and permission-role assignments. The goal is to find a role model, which best leverages the RBAC model by satisfying the business, security and regulatory drivers: Lowering the costs for the administration of access control and equip employees with necessary permissions such that they can efficiently full-fill their tasks while keeping security requirements and being compliant.
    How this goal is measured in the role model has been described by several quality criteria\cite{Kunz}\cite{Frank}. The criteria concern the role model state, individual roles or both. In the following the different criteria is summarized into three different categories.
    \begin{itemize}
        \item \textbf{Completeness (or Confidentiality/Accessibility Violations)}\\
        By completeness it is meant to rebuild the current user-permission assignment state by the role model. When the user-role and the permission-role assignments of the role model are resolved, it should cover all current user-permission assignments.\\
        It is assumed that the current user-permission assignments are in a state, where users get the necessary access to efficiently perform their tasks and no security or compliance regulations are violated. This of course is an ideal situation, but in reality the current state often has quite some noise, especially when no IAM solution has been in place before. Users tend to have more permissions than they actually need, since it is unlikely that someone will claim that he has too many permissions. The measure of completeness of the role model state is therefore in accordance to the initial access control configuration.\\
        A certain amount of overentitlements (users get too many permissions) or underentitlements (users get too few permissions) in the access control policies can be acceptable, if the least privilege principle is too costly to implement in practice.\\
        The combinations of permissions within roles or the combination of user-role assignments might violate security regulations such as SoD. This measure is therefore not only taking individual roles but also the role model state into account.\\
        Constraints could also be ensured in a mechanism posterior to the role model, where individual permissions are detracted from users again, if it violates a constraint. Allowing constraint violations in the role model increases not only the processing and reliance on the posterior mechanism but also the auditing effort.
        \item \textbf{Complexity (or Number of Roles and Assignments)}\\
        The complexity of a role model is measured in its number of roles and the number of user-role and permission-role assignments. It is often connected to the maintenance costs of a role model.\\
        The more roles the role model has, the more maintenance effort is expected. Hence, a minimal set of roles is preferable. Furthermore it should be obvious that the total number of roles should be smaller than the total number of users or total number of permissions. Otherwise there would be no use of the advantage of having RBAC in comparison to IBAC in terms of administration costs. When each user has its individual role with the according individual bundle of permissions, the abstraction layer of the role becomes obsolete.\\
        Also the more user-role and permission-role assignments are needed, the more maintenance effort is expected. Large roles with many permissions may reduce the number of user-role assignments, but may lead to more confidentiality/accessibility violations (conflicting Completeness). The same applies for if each role is used by many users. Small roles with few permissions on the other hand can lead to more administration effort as mentioned above, since many roles are necessary for achieving completeness. The same applies if each role is only used by very few users.\\
        A role model probably consists of very general large roles and specialized small roles. Determining a fix boundary of how many permissions or users can be assigned to a role requires knowledge of the role model or is given by company or security regulations.
        \item \textbf{Comprehension}\\
        A recently more discussed topic is the "meaningfulness" of roles \cite{Xu}\cite{Frank}. It is important that the administrators, which are maintaining the role model, can logically understand the role model for maintaining the roles and assignments confidently. Otherwise it might happen that they avoid to work with the role model since they feel not confident to stay in line with security and compliance regulations. Or it will cause them extra effort and costs to work with the role model. The roles should be therefore comprehensive, which can be achieved by giving them a meaning close to business roles, e.g. a role "Employee", which contains all permissions every employee will get and is assigned to every employee-user.\\
        This criteria can loosen some of the other criteria, which rather concentrate on the compression of the access control information \cite{Frank2013}. For achieving more intuitive meaningful roles it might be necessary to allow more roles than the minimal number of roles resulting from compression. More roles might result in more assignments, which are necessary to keep the role model more comprehensive. Lowering costs by having a more comprehensive role model may contradict lowering costs by having a less complex role model.
    \end{itemize}
    
    \subsection{Role Engineering}
    Role engineering \cite{coyne2011role} describes the process to create a role model for RBAC. This task is proved to be very difficult in large enterprises.\\
    There are three different approaches to conquer the goal of finding the right role model: Top-Down, Bottom-Up and the Hybrid approach\cite{coyne2011role}\cite{Frank}.
    \begin{itemize}
        \item \textbf{Top-Down Approach}\\
        One approach is to do a top-down analysis, where the roles are built out of business information. Business processes, business roles and security policies are analysed to build a suitable role model. The resulting roles contain high-level permissions, which need to be mapped to technical permissions that are used by IT systems. The roles are easy to understand as they are derived from business concepts. The analysis of the business information by experts to a high-level role model and the mapping into low-level accesses by IT Specialists are very time-consuming and costly. Furthermore the resulting role model could lead to a different access control configuration than the current one. It is likely that users get less permissions than they used to have, which might prevent them of doing their tasks efficiently.
        \item \textbf{Bottom-Up Approach}\\
        The bottom-up approach exploits the current user-permission assignments and tries to gather a role model out of it. Since this approach often uses Data Mining Techniques, the method is also called Role Mining\cite{Kuhlmann}. This approach on the other hand is often failing in generating a comprehensive role model, which is accepted by the administrators\cite{Frank2013}.
        \item \textbf{Hybrid approach}\\
        Since the advantages and disadvantages of the top-down and the bottom-up approach are mirrored, hybrid approaches have been suggested \cite{Frank}\cite{6274146}. In these approaches the business information is leveraged to guide the computational generation of a role model out of user-permission assignments.
    \end{itemize}
    
    \subsection{Role Mining Problem Definitions}
    There are several role-mining problem definitions existing. The problem definitions, which will be considered in this thesis, are introduced in the following.
    \begin{itemize}
        \item \textbf{The Basic Role-Mining Problem}\\
        \cite{Vaidya:2007}
        \item \textbf{The $\delta$-approx Role-Mining Problem}\\
        \cite{Vaidya:2007}
        \item \textbf{The Min-Noise Role-Mining Problem}\\
        \cite{Vaidya:2007}
        \item \textbf{The Min-Edge Role-Mining Problem}\\
        \cite{4497438}
        \item \textbf{The Interference Role-Mining Problem}\\
        \cite{Frank:2013}
    \end{itemize}
    The role-mining problems have related problems like e.g. the Boolean Decomposition Problem (BDM) or the Tiling Problem.